星期一到星期三满课成就达成,老板口中二十年没见过掉轮
子二个星期掉了二次成就达成。。数据库有点难,范式,分
解算法,多表查询,回想梅子溪说什么半个小时搞定真的是
搞笑,对知识还是要多一点敬畏啊小伙砸。今晚听了文澜学
院那个经济学前沿方法,就是院长请一些老师讲研究领域和
洗脑之大学生活如何度过系列,社会调查,计量经济学的重
性算是不多的收获,不过老师讲的关于社会调查时有村民以
为是上级下访而对他们寄予厚望,渴望解决当地领导作威作
福、生活苦不堪言的问题,却得知只是普通大学生的失望心
情。而老师一行也深受触动,产生阅书十数载空有满肚学识
却对民生疾苦无能为力的迷茫。我觉得这是一个值得深思的
故事,可能人生每个阶段的解读都会截然不同。现在的我觉
得这是一个价值观的问题,之前我有一个属于自己的理论,
人与人之间唯一可比的东西,衡量因素不是有钱与没钱,长
寿与短命,功成名就与默默无闻,甚至也不是没有衡量因素,
我觉得是一辈子所积累的效用,数值化来说就是从有意识到
死亡的每一年每一天做的每一件事效用(可正可负)之和,
如果所做的事情具有延续性的话就改成期望效用,对于期望
的理解可以从一个例子来说明,假如打游戏的效用是50,然
而第二天要考试,因为打游戏会延续到考试影响考试的成绩,
进而延续到出成绩那天对你的效用,假设打游戏会让你少考
10分,而少10分给你带来的效用是-200,因此考试之前打
游戏这件事情的期望效用就是-150.同理,假设谈恋爱期间
恋人给你带来的效用是10000,失恋及相当长的失恋期(随
着时间的流逝负效用会呈递减趋势)给你带来的效用是-200
00(算法应该类似于定积分),所以谈恋爱以及失恋这件延
续性的事件给你带来负效用。如果失恋带来的负效用不是-2
0000而是-5000甚至是+5000呢,那么谈恋爱这个事件就具
正效用的效应,为什么分析这二种情况,因为这个简单的理
论我觉得在信息完全(比如知道在谈恋爱之前知道失恋之后
的负效用是多少)的情况下可以决定到世界上所有的决策。
————赵尚琦的效用理论1。扯这么多回到对于前面所提问题
的分析,关于价值观,关于人活着的价值衡量,我觉得同样
可以用效用论来分析。价值我觉得是具有社会属性的,一个
人的价值是别人眼中自己的价值,定量来说你对人类社会所
有的人带来的价值之和,给别人带来的价值,就可以用给别
人带来的效用总和来衡量,同样的,如果是持续性的事件就
是期望效用的总和。煮个栗子,袁隆平好吧,给全世界所有
的人带来了效用,等等作出巨大贡献的科学家,堪称人类社
会的MVP。第二个栗子,还是关于谈恋爱,假如谈恋爱给你
带来负效用-10000,而这个负效用是你对象一个人造成的,
所以在计算你对象的总价值的时候,需要减去这个10000.
—————赵尚琦的效用理论2 根据理论2,来分析老师的迷茫,
仿佛浅显易懂,很明显,他的迷茫来自于未能对村民产生价
值,进而对自己的总价值产生了怀疑,而却忽略了作为老师,
直接影响的是学生,老师对社会上所有人的总价值大部分应该
来自学生,或者是研究成果对人类的贡献,这才是他的价值所
在。


