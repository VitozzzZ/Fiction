晚上自习间没事Google了一下下操作系统老师屈振新。我一直
觉得,上课让人听不懂的老师有二种,一种是像数学分析老师那
样,笑着对同桌说以我的智商听不懂的话绝对是老师的问题,事
实也确实是这样;第二种就是屈老师,让我想起了王亚,语速思
维极快,知识点源源不绝,以致于即使听不懂也只能一言不发仔
细回味,说的许多话比如“对待数据结构我们要做到随~心~所~欲”
“进程是对CPU的抽象,地址空间是对内存的抽象,文件是对磁盘
的抽象”等等已成为心中的经典。Google之后倒没有什么很八卦
的发现(比如之前Google复旦谢启鸿居然发现有出轨离婚的法庭
记录。。),倒是因为一张屈老师的实验报告顺藤摸瓜进了一个学
长的博客,可以说是十分斯国一咧,从10年坚持写到今天,置顶上
面写到已有49万余字,内容无非是高中大学的一些回忆,人生的选
择,单身程序员脑补日常等等,但是娓娓道来之间居然有让我一直
读下去的欲望。可能因为他是学长,但更重要的原因我想是走进别
人内心世界而一览无余的快感,就像一本忘记上锁的日记摆在面前。
不由得想到了一句话,迟早有一天会有一个人翻完你过去所有的朋
友圈所有的说说,并笑着对你说我愿意弥补我没有出现在你生命里
那缺失的部分。但是票圈说说有多少无病呻吟说给别人看的假话心
里都有b数的吧,日志可以说是挺完美的表达了。心真的静了不少呢。
