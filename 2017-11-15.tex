晚上自习间没事Google了一下下操作系统老师屈振新。我一
直觉得,上课让人听不懂的老师有二种,一种是像数学分析
老师那样,笑着对同桌说以我的智商听不懂的话绝对是老师
的问题,事实也确实是这样;第二种就是屈老师,让我想起
了王亚,语速思维极快,知识点源源不绝,以致于即使听不
懂也只能一言不发仔细回味,说的许多话比如“对待数据结
构我们要做到随~心~所~欲”“进程是对CPU的抽象,地址空
间是对内存的抽象,文件是对磁盘的抽象”等等已成为心中
的经典。Google之后倒没有什么很八卦的发现(比如之前
Google复旦谢启鸿居然发现有出轨离婚的法庭记录。。),
倒是因为一张屈老师的实验报告顺藤摸瓜进了一个学长的博
客,可以说是十分斯国一咧,从10年坚持写到今天,置顶上
面写到已有49万余字,内容无非是高中大学的一些回忆,人
生的选择,单身程序员脑补日常等等,但是娓娓道来之间居
然有让我一直读下去的欲望。可能因为他是学长,但更重要
的原因我想是走进别人内心世界而一览无余的快感,就像一
本忘记上锁的日记摆在面前。不由得想到了一句话,迟早有
一天会有一个人翻完你过去所有的朋友圈所有的说说,并笑
着对你说我愿意弥补我没有出现在你生命里那缺失的部分。
但是票圈说说有多少无病呻吟说给别人看的假话心里都有b数
的吧,日志可以说是挺完美的表达了。心真的静了不少呢。
