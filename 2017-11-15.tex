晚上自习间没事Google了一下下操作系统老师屈振新。我一直觉得,上课让人听不懂的老师有二种,一种是像数学分析老师那样,笑着对同桌说以我的智商听不懂的话绝对
是老师的问题,事实也确实是这样;第二种就是屈老师,让我想起了王亚,语速思维极快,知识点源源不绝,以致于即使听不懂也只能一言不发仔细回味,说的许多话比如“
对待数据结构我们要做到随~心~所~欲”“进程是对CPU的抽象,地址空间是对内存的抽象,文件是对磁盘的抽象”等等已成为心中的经典。Google之后倒没有什么很八卦的发
现(比如之前Google复旦谢启鸿居然发现有出轨离婚的法庭记录。。),倒是因为一张屈老师的实验报告顺藤摸瓜进了一个学长的博客,可以说是十分斯国一咧,从10年坚
持写到今天,置顶上面写到已有49万余字,内容无非是高中大学的一些回忆,人生的选择,单身程序员脑补日常等等,但是娓娓道来之间居然有让我一直读下去的欲望。可能
因为他是学长,但更重要的原因我想是走进别人内心世界而一览无余的快感,就像一本忘记上锁的日记摆在面前。不由得想到了一句话,迟早有一天会有一个人翻完你过去所
有的朋友圈所有的说说,并笑着对你说我愿意了解你的
